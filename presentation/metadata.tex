\begin{frame}
	\frametitle{Metadata format}
	\setbeamertemplate{description item}[align left]
	\begin{description}[finish\_time]
		\item[id] 
			Server-side unique identification of the video.
		\item[name]
			File name in client's local file system.			
		\item[finish\_time]
			Video creation time in the format: yyyy-mm-dd hh:mm:ss
		\item[duration]
			Video duration in milliseconds.	
		\item[width]
			Video frame width in pixels.
		\item[height]
			Video frame height in pixels.		
		\item[shaking]
			Amount of shaking detected by sensors.
		\item[tilt]
			The amount of tilt holding of the camera.
		\item[status]
			Video status. Indicates video life cycle phase.
		\item[event\_id]
			Unique server-side identifier of an event.		
	\end{description}
\end{frame}

\begin{frame}[fragile]
	\frametitle{JSON vs. XML}
		\begin{itemize}
			\item JSON more lightweight then  XML, less size overhead.
			\item Library org.json for Java for fast parsing.
			\item Serialization and deserialization in Json more efficient then XML.
			\item Describes meta data properly, XMl better used for data in the document paradigm
		\end{itemize}
\end{frame}

\begin{frame}[fragile]
	\frametitle{Metadata example (formatted) }
		HTTP\_POST /event/\textit{id}\footnotemark
		\begin{verbatim}
		{
		    "id":30,
		    "duration":3669,
		    "height":1080,
		    "shaking":0,
		    "tilt":24,
		    "width":1920,
		    "finish_time":"2014-11-05 14:01:12.0",
		    "name":"VID_20141105_140107.mp4"
		}
		\end{verbatim}
		\footnotetext{\textsl{event\_id} is transfered as a part of URL}
\end{frame}
