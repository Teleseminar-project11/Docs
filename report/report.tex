\documentclass[conference]{IEEEtran}

\usepackage{cite}
\usepackage[pdftex]{graphicx}
% declare the path(s) where your graphic files are
\graphicspath{{../images/}}
\usepackage[cmex10]{amsmath}
\usepackage{algorithmic}
%\usepackage{array}
\usepackage[caption=false,font=footnotesize]{subfig}
\usepackage{stfloats}
\usepackage{url}
\usepackage[utf8]{inputenc}
\usepackage{multirow}
\usepackage{booktabs}
\usepackage{tabularx}
\newcolumntype{R}{>{\raggedleft\arraybackslash}X}%

% correct bad hyphenation here
\hyphenation{op-tical net-works semi-conduc-tor}

\begin{document}

\title{Automatic Mobile Video Director}

\author{
	\IEEEauthorblockN{Alexander Egurnov}
	\IEEEauthorblockA{University of Mannheim\\
		aegurnov@mail.uni-mannheim.de}
\and
	\IEEEauthorblockN{Thilo Weigold}
	\IEEEauthorblockA{University of Mannheim\\
		tweigold@mail.uni-mannheim.de}
\and
	\IEEEauthorblockN{Jon Pettersen}
	\IEEEauthorblockA{University of Oslo\\
		jonup@student.matnat.uio.no}
\and
	\IEEEauthorblockN{Alf-André Walla}
	\IEEEauthorblockA{University of Oslo\\
		alfandrw@ifi.uio.no}
}

% This a recommended way to give more than 3 authors but it looks ugly.
% Default author block above still fits in one line.
%\author{
	%\IEEEauthorblockN{
		%Alexander Egurnov\IEEEauthorrefmark{1},
		%Thilo Weigold\IEEEauthorrefmark{2},
		%Jon Pettersen\IEEEauthorrefmark{3}, 
		%Alf-André Walla\IEEEauthorrefmark{4}
	%}
	%\IEEEauthorblockA{
		%\IEEEauthorrefmark{1}University of Mannheim\\
		%aegurnov@mail.uni-mannheim.de
	%}
	%\IEEEauthorblockA{
		%\IEEEauthorrefmark{2}University of Mannheim\\
		%tweigold@mail.uni-mannheim.de
	%}
	%\IEEEauthorblockA{
		%\IEEEauthorrefmark{3}University of Oslo\\
		%jonup@student.matnat.uio.no
	%}
	%\IEEEauthorblockA{
		%\IEEEauthorrefmark{4}University of Oslo\\
		%alfandrw@ifi.uio.no
	%}
%}

\maketitle

\begin{abstract}
The abstract goes here.
\end{abstract}

\section{Introduction}

This demo file is intended to serve as a ``starter file''
for IEEE conference papers produced under \LaTeX\ using
IEEEtran.cls version 1.7 and later.
% You must have at least 2 lines in the paragraph with the drop letter
% (should never be an issue)
I wish you the best of success.

% An example of a floating figure using the graphicx package.
% Note that \label must occur AFTER (or within) \caption.
% For figures, \caption should occur after the \includegraphics.
% Note that IEEEtran v1.7 and later has special internal code that
% is designed to preserve the operation of \label within \caption
% even when the captionsoff option is in effect. However, because
% of issues like this, it may be the safest practice to put all your
% \label just after \caption rather than within \caption{}.
%
% Reminder: the "draftcls" or "draftclsnofoot", not "draft", class
% option should be used if it is desired that the figures are to be
% displayed while in draft mode.
%
%\begin{figure}[!t]
%\centering
%\includegraphics[width=2.5in]{myfigure}
% where an .eps filename suffix will be assumed under latex, 
% and a .pdf suffix will be assumed for pdflatex; or what has been declared
% via \DeclareGraphicsExtensions.
%\caption{Simulation Results.}
%\label{fig_sim}
%\end{figure}

% Note that IEEE typically puts floats only at the top, even when this
% results in a large percentage of a column being occupied by floats.
% However, the Computer Society has been known to put floats at the bottom.

% An example of a double column floating figure using two subfigures.
% (The subfig.sty package must be loaded for this to work.)
% The subfigure \label commands are set within each subfloat command,
% and the \label for the overall figure must come after \caption.
% \hfil is used as a separator to get equal spacing.
% Watch out that the combined width of all the subfigures on a 
% line do not exceed the text width or a line break will occur.
%
%\begin{figure*}[!t]
%\centering
%\subfloat[Case I]{\includegraphics[width=2.5in]{box}%
%\label{fig_first_case}}
%\hfil
%\subfloat[Case II]{\includegraphics[width=2.5in]{box}%
%\label{fig_second_case}}
%\caption{Simulation results.}
%\label{fig_sim}
%\end{figure*}
%
% Note that often IEEE papers with subfigures do not employ subfigure
% captions (using the optional argument to \subfloat[]), but instead will
% reference/describe all of them (a), (b), etc., within the main caption.

% An example of a floating table. Note that, for IEEE style tables, the 
% \caption command should come BEFORE the table. Table text will default to
% \footnotesize as IEEE normally uses this smaller font for tables.
% The \label must come after \caption as always.
%
%\begin{table}[!t]
%% increase table row spacing, adjust to taste
%\renewcommand{\arraystretch}{1.3}
% if using array.sty, it might be a good idea to tweak the value of
% \extrarowheight as needed to properly center the text within the cells
%\caption{An Example of a Table}
%\label{table_example}
%\centering
%% Some packages, such as MDW tools, offer better commands for making tables
%% than the plain LaTeX2e tabular which is used here.
%\begin{tabular}{|c||c|}
%\hline
%One & Two\\
%\hline
%Three & Four\\
%\hline
%\end{tabular}
%\end{table}

% Note that IEEE does not put floats in the very first column - or typically
% anywhere on the first page for that matter. Also, in-text middle ("here")
% positioning is not used. Most IEEE journals use top floats exclusively.
% However, Computer Society journals sometimes do use bottom floats - bear
% this in mind when choosing appropriate optional arguments for the
% figure/table environments.

\section{Related work}
Describe articles and how our work differs from theirs. 
Throw in some references \cite{shrestha_automatic_2010} so bibliography does not look empty.
\cite{seshadri_demand_2014}

\section{Methodology}

\subsection{System Overview}

\subsection{Client application}

\subsection{Server application}
Server general description goes here.
Video storage, database connection, server framework description.

\subsection{Client-server interaction}

\subsubsection{Protocols}
Our Automatic Mobile Video Director server implementation provides 
a general interface to applications which wish to interact with it. 
It is implemented through HTTP requests to certain server locations result.

\begin{description}
	\item[GET /events]\hfill\\
		Lists all events (including videos) in~JSON.
		
	\item[GET /event/\textit{id}]\hfill\\
		Returns Event (including videos) in~JSON.
				
	\item[POST /event/new]\hfill\\
		Create new event from JSON.
		Expects request body to be a JSON string containing attribute~\textit{name}.
		
	\item[POST /event/\textit{id}]\hfill\\
		Upload JSON metadata about a video for Event with given \textit{id}.
		
	\item[PUT /video/\textit{video\_id}]\hfill\\
		Upload video~\textit{video\_id} from Event~\textit{id}.
		Expects request body to be a file stream containing a full video file.
		
	\item[GET /video/\textit{video\_id}]\hfill\\
		Retrieve video~\textit{video\_id} from Event~\textit{id}.
		
	\item[GET /selected]\hfill\\
		Retrieve a list of selected but not yet uploaded videos in JSON.		
		
\end{description}

\subsection{Metadata format}
JSON vs XML arguments here
As a final result we should state that metadata is transferred in JSON format.
% TODO Dummy metadata description.
\begin{description}
	\item[id]\hfill\\
		Client-side unique identification of the video.
		
	\item[filename]\hfill\\
		File name in client's local file system.
		
	\item[timestamp]\hfill\\
		Video creation time.
		%TODO Is it really so?
		
	\item[duration]\hfill\\
		Video duration in frames.
		%TODO This thing too.
		
	\item[width]\hfill\\
		Video frame width in pixels.

	\item[height]\hfill\\
		Video frame height in pixels.		
	
	\item[shaking]\hfill\\
		Amount of shaking detected by sensors.
	
	\item[status]\hfill\\
		Video status. Indicates video life cycle phase.
		% TODO It would be nice to decide on video lifecycle both client- and server-side.
		%	It would make a nice diagram to include in the report.
	
	\item[serverId]\hfill\\
		Server-side unique identification of the video. Needed for coordination of all clients.
		
\end{description}

\section{Video Director Algorithm}

\subsection{Video life cycle}

\subsection{Selection algorithm}

\section{Evaluation}
How good/bad it is.

\subsection{Data Traffic}

\subsection{Battry consumptio}
Who wants to test it?

\subsection{Selection criteria}

\section{Future Work}
Put down all the awesome ideas we have.

\section{Conclusion}
The conclusion goes here.


% trigger a \newpage just before the given reference
% number - used to balance the columns on the last page
% adjust value as needed - may need to be readjusted if
% the document is modified later
%\IEEEtriggeratref{8}
% The "triggered" command can be changed if desired:
%\IEEEtriggercmd{\enlargethispage{-5in}}

% trigger a \newpage just before the given reference
% number - used to balance the columns on the last page
% adjust value as needed - may need to be readjusted if
% the document is modified later
%\IEEEtriggeratref{8}
% The "triggered" command can be changed if desired:
%\IEEEtriggercmd{\enlargethispage{-5in}}

% references section

% can use a bibliography generated by BibTeX as a .bbl file
% BibTeX documentation can be easily obtained at:
% http://www.ctan.org/tex-archive/biblio/bibtex/contrib/doc/
% The IEEEtran BibTeX style support page is at:
% http://www.michaelshell.org/tex/ieeetran/bibtex/
\bibliographystyle{IEEEtran}
\bibliography{report}
% argument is your BibTeX string definitions and bibliography database(s)
%\bibliography{IEEEabrv,../bib/paper}

%\newpage

\begin{table*}[t]
	\centering
	\renewcommand{\arraystretch}{1.5}
	\caption{Task distribution}
	\label{tab:task_distr}
	\begin{tabular}{lllr}
		\toprule
		Part & Task  & Subtask & Responsible \\
		\midrule
		Android application 
			& Video Capture &       & Thilo Weigold \\
			& Sensor data collection &       & Thilo Weigold \\
			& Metadata class &       & Thilo Weigold \\
			& Http Client & Post Method & Thilo Weigold \\
			&       & Cookies, Callbacks & Alexander Egurnov \\
			&       & Get \& Update methods & Alexander Egurnov \\
			& Background upload service  &       & Thilo Weigold \\
			& SQLite database connection &       & Thilo Weigold \\
			& Preferences &       & Alexander Egurnov \\
			& GUI   &       & Thilo Weigold, Alexander Egurnov \\
			& Client-server data exchange &       & Alexander Egurnov \\
		\midrule
			Server application 
			& RESTful Server application & Client authorization & Alf-André Walla \\
			&       & Request processing & Alf-André Walla \\
			&       & Video and Event logic & Alf-André Walla \\
			& MySQL database connection &       & Jon Pettersen \\
			& Video Director  &       & Alf-André Walla \\
			& Client-server data exchange debug &       & Alexander Egurnov \\
			& Video upload &       & Alf-André Walla, Alexander Egurnov \\
		\midrule
			Web Server 
			& MySQL Administration &       & Alexander Egurnov \\
			& Nginx setup for streaming video &       & Alexander Egurnov \\
			&       &       &  \\
			&       &       &  \\
		\midrule
			Documentation & Basic template formatting & & Alexander Egurnov \\
			&       &       &  \\
			&       &       &  \\
			&       &       &  \\
		\bottomrule
    \end{tabular}%
\end{table*}%
\hfill


\end{document}
